
\vspace{0.5 cm}

\noindent\fbox{
    \parbox{\textwidth}{
        You may not modify your test during scanning.  Please do not have any pen or pencil out during scanning.  If you need to make something darker, consult with a proctor.  If you need to leave the scanning area (e.g., to go to the restroom) leave your test with the  proctor.
    }
}

\vfill
%%%%%%%%%%%%%%%%%%%%%%%%%%%%%%%%%%%%%%%%%%%%%%%%%%%%%
\centerline{\bf Honor Pledge}  % and Submission Instructions}

\bigskip

We appreciate that being a student can be stressful. You may be tempted to cheat in order to improve a grade or otherwise advance your career. You might do these things and potentially not get caught. However, if you cheat, no matter how much you may have learned in this class, you have failed to learn the most important lesson of all. All relationships, including this one, are founded on trust. And trust is hard to earn, and easy to lose. \emph{If you're struggling with the material, let us know.} We can help. However, we take academic integrity very seriously. Any suspected cases of cheating will be immediately filed with the Department of Student Rights and Community Standards.  \footnote[1]{The first part of this paragraph is ``borrowed'' from \url{https://astro.berkeley.edu/prospective-students/integrity-statement/}}


\vspace{0.5 cm}

\emph{By signing this pledge, I affirm that this work is entirely my own work, and I have not consulted any sources or persons not explicitly allowed by my instructor.}

\vspace{0.5 cm}

\textbf{Sign below to acknowledge that you have read this pledge.} % If you're writing on your own paper (without a printed copy of the exam), copy the above \emph{sentence} and sign below it on your \emph{first page}.}

\vspace{0.5 cm}

Name: \underline{\hspace{10 cm}}  

{\bf Please write full name clearly so Gradescope can identify your test.  Thanks!}

\vspace{0.5 cm}

Signature: \underline{\hspace{10 cm}}


\vfill
\eject

{\bf Submitting Your Work:}
\begin{itemize}
%	\item You are allowed to use a Study Sheet: one 8.5x11'' sheet of paper with handwritten notes (both sides is ok).  You may not use any other books, notes, or calculators.  You may not collaborate with others.

%	\item Each problem will be graded as credit/no credit, but you will have another chance to receive credit on a similar problem on the next FA. The outcome(s) satisfied by each problem are listed at the top and bottom of the page. Remember that once you have mastered an outcome (i.e., received credit for an outcome on two different Friday Assessments), you no longer need to do those problems.  %those problems will no longer show up on your personalized test.

%	\item If you're testing online: 
%		\begin{itemize}
%			\item Do not leave the call until your exam is submitted on Gradescope. 

			\item When you have finished  (or when exam time is over) bring your test and all belongings to the scanning area and scan your work as a \textbf{single pdf} to submit.  
	
				{\bf We strongly recommend using the Gradescope app to scan.}
	
			\item Submit the exam  in Gradescope (\url{www.gradescope.com}).
			
%				You can use a free scanner like: \url{https://acrobat.adobe.com/us/en/mobile/scanner-app.html}

			\item {\bf Important: Assign pages for each problem (after submitting).}
			
				Select the previous page (with your outcomes mastery at the top) for any problems you have already mastered.
				
			\item Leave your paper test with a proctor and show them the submission confirmation.  
			
			\item Have a great weekend!

%		\end{itemize}

%	\item \textbf{If you experience any technical issues}, please email \emph{both} your instructor \emph{and} the course coordinator Becci Torrey at rtorrey@brandeis.edu or Keith Merrill at merrill2@brandeis.edu.
	\end{itemize}


{\bf Important Reminders:}
\begin{itemize}
	\item {\bf Show all your work!}  We are looking for clear communication of the \emph{process}.  A correct answer with insufficient work might not earn credit.

	\item For all word problems:
		\begin{itemize}
			\item Include units in your answer.
			\item Add a concluding sentence. For example, if the problem says find the height of the wall, write ``The wall is 12ft tall."
		\end{itemize}
	\item For all graphs:
		\begin{itemize}
			\item Label at least three points on the graph.  (Put a dot there and write the $x$- and $y$- coordinates next to it.)
			\item Make sure all important features ($x$- and $y$-intercepts, increasing/decreasing, positive/negative, the way the graph curves, etc.) are  clear.  
%%			\item Labeling (label at least three points) % and your axes)
%%			\item Legibility (make details such as increasing/decreasing, positive/negative unambiguous)
%%			\item Large-ness (there's no such thing as too big of a graph)
		\end{itemize}
	\end{itemize}
	
	
{\bf Formulas:}
\begin{itemize}
    \item Quadratic Formula: $\displaystyle x = \frac{-b \pm \sqrt{b^2 - 4ac}}{2a}$
    \item distance $=$ rate $\times$ time
    \item Pythagorean Theorem: $\displaystyle a^2 + b^2 = c^2$
    \item Compound Interest: $\displaystyle A(t) = P\left ( 1 + \frac rn \right )^{nt}$
    \item Volume of a cylinder: $\displaystyle V = \pi r^2 h$
    \item Surface area of a cylinder: $\displaystyle A = 2\pi r^2 + 2\pi r h$
\end{itemize}
	
	
\vfill
\eject 
